\documentclass{article}

% Packages for code listing and syntax highlighting
\usepackage{listings}
\usepackage{xcolor}
\usepackage[margin=3cm]{geometry} % Adjust the margin value as desired
\usepackage{setspace}
\usepackage{tikz}
\usepackage{graphicx}
\usepackage{float}
\usepackage{textcomp}
\usepackage{multicol}
\usepackage{enumitem}

\onehalfspacing

% Define the color theme
\definecolor{codebackground}{RGB}{242, 242, 242}
\definecolor{codekeyword}{RGB}{0, 0, 255}
\definecolor{codecomment}{RGB}{63, 127, 95}
\definecolor{codestring}{RGB}{163, 21, 21}

% Code listing style for all languages
\lstdefinestyle{mystyle}{
    backgroundcolor=\color{codebackground},
    basicstyle=\footnotesize\ttfamily,
    keywordstyle=\color{codekeyword}\bfseries,
    commentstyle=\color{codecomment}\itshape,
    stringstyle=\color{codestring},
    numbers=left,
    numberstyle=\tiny\color{codecomment},
    stepnumber=1,
    numbersep=8pt,
    showstringspaces=false,
    breaklines=true,
    frame=single,
    frameround=none,
    framesep=5pt,
    rulecolor=\color{codebackground},
    tabsize=4,
    captionpos=b,
    xleftmargin=15pt,
    xrightmargin=15pt
}

% Set the default style for code listings
\lstset{style=mystyle}

% Additional packages and settings for math typesetting
\usepackage{amsmath}
\usepackage{amssymb}
\usepackage{bm}

% Define your document content
\begin{document}

\title{Raising a Complex Number to A Power Using De Moivre's Theorem}
\author{Abyan Majid}
\date{\today}
\maketitle
\begin{center}
De Moivre's Theorem gives us a way to raise any complex number $z$ to any integer power $n$.

    Specifically, De Moivre's Theorem is defined as:\\
    \fbox{$z^n=r^n(\cos(n\theta)+i\sin(n\theta))$, for $n\in\mathbb{Z}$}

\vspace{\baselineskip}

\noindent \textbf{De Moivre's theorem actually also applies to real and complex powers}, but such case is rare.

\vspace{\baselineskip}

\noindent Example: Given that $z=3(\cos\frac{\pi}{4}+i\sin\frac{\pi}{4})$, find $z^4$
    $$z^4=3^4[\cos(4)(\frac{\pi}{4})+i\sin(4)(\frac{\pi}{4})]$$
    $$z^4=81(\cos\pi+i\sin\pi)$$
    $$z^4=-81$$



You can of course apply De Moivre's Theorem in the exponential form, like so: \fbox{$z=r^ne^{in\theta}$}

\vspace{\baselineskip}

\noindent(Btw, pronounce "De Moivre" as "De Moave" because it's french)

\end{center}

\end{document}