\documentclass{article}

% Packages for code listing and syntax highlighting
\usepackage{listings}
\usepackage{xcolor}
\usepackage[margin=3cm]{geometry} % Adjust the margin value as desired
\usepackage{setspace}
\usepackage{tikz}
\usepackage{graphicx}
\usepackage{float}
\usepackage{textcomp}
\usepackage{multicol}
\usepackage{enumitem}

\onehalfspacing

% Define the color theme
\definecolor{codebackground}{RGB}{242, 242, 242}
\definecolor{codekeyword}{RGB}{0, 0, 255}
\definecolor{codecomment}{RGB}{63, 127, 95}
\definecolor{codestring}{RGB}{163, 21, 21}

% Code listing style for all languages
\lstdefinestyle{mystyle}{
    backgroundcolor=\color{codebackground},
    basicstyle=\footnotesize\ttfamily,
    keywordstyle=\color{codekeyword}\bfseries,
    commentstyle=\color{codecomment}\itshape,
    stringstyle=\color{codestring},
    numbers=left,
    numberstyle=\tiny\color{codecomment},
    stepnumber=1,
    numbersep=8pt,
    showstringspaces=false,
    breaklines=true,
    frame=single,
    frameround=none,
    framesep=5pt,
    rulecolor=\color{codebackground},
    tabsize=4,
    captionpos=b,
    xleftmargin=15pt,
    xrightmargin=15pt
}

% Set the default style for code listings
\lstset{style=mystyle}

% Additional packages and settings for math typesetting
\usepackage{amsmath}
\usepackage{amssymb}
\usepackage{bm}

% Define your document content
\begin{document}

\title{Roots and Complex Roots}
\author{Abyan Majid}
\date{\today}
\maketitle

\noindent Recall that for a positive real number $a$ and a positive integer $n$, the $n$th root of $a$, denoted as $\sqrt[n]{a}$ or $a^{1/n}$, is the positive real number $x$ such that $x^n=a$.

\vspace{\baselineskip}

\noindent For example, \\ $\sqrt{9}=3$ \\ $3^2=9$

\section{Roots of complex numbers}

\begin{center}
    Any complex number except 0 has exactly $n$ distinct $n$th roots.
\end{center}
To find the $n$th roots of a complex number, you first add $2k\pi$ to $\theta$ in order to take account of all possible arguments, where $k$ is an integer multiplier of $2\pi$ (this is all given the fact that any complex number has infinitely many arguments of integer multiples $2\pi$).

\begin{center}
    So, we now have $z=r[\cos(\theta+2k\pi)+i\sin(\theta+2k\pi)]$, or $z=re^{i(\theta+2k\pi)}$.
\end{center}

\noindent From here, we can actually use the De Moivre's Theorem in REVERSE in order to find roots! So, if in the usual case of $z^n$, we would do,
$$z^n=r^n(\cos n\theta + i\sin n\theta) = r^ne^{in\theta}$$

\noindent in order to find the roots, i.e. $z^{1/n}$, we can instead do:

$$z^{1/n}=r^{1/n}[\cos(\frac{\theta+2k\pi}{n})+i\sin(\frac{\theta+2k\pi}{n})] = r^{1/n}e^{i(\frac{\theta+2k\pi}{n})}$$

\begin{center}
    This is the form of De Moivre's theorem you would use to find the $n$th roots of complex numbers.
\end{center}

\noindent So when raising a complex number to a power using De Moivre's theorem, we raise the modulus $r$ to the power of $n$ and multiply the argument $\theta$ by $n$. ON THE OTHER HAND, when finding the ROOTS of a complex number, we can take the $n$th root of the modulus $r$ and divide all arguments by $n$. You know, the opposite of raising to a power is taking a root, and the opposite of multiplying is dividing. 

\section{Complex roots of polynomials}

Polynomials can have complex roots, and you should give these complex roots in the event that you're asked to give the roots "over the set of complex numbers". If you're NOT asked to provide complex roots, there is no need to do so.

\vspace{\baselineskip}

\noindent To get the complex roots of quadratics, you may use the quadratic formula. Otherwise, you can do factorization to get the complex roots of polynomials of degree greater than 2.

\begin{center}
    Example: Find the roots of $x^3-x^2+x-1$
    $$ x^3-x^2+x-1=0 $$
    We can factor out $x-1$,
    $$ (x-1)(x^2+1) = 0$$
    Using the zero factor principle, ie. if $ab=0$ then $a=0$, $b=0$,
    \begin{multicols}{2}
        \section*{}
        \vspace{-\baselineskip} % Erases space of section
        $$x-1=0$$
        $$x=1$$
    
        \section*{}
        \vspace{-\baselineskip} % Erases space of section
        $$x^2+1=0$$
        $$x=\pm\sqrt{-1}=\pm i$$
    \end{multicols}
    \noindent \fbox{$x=1,i,-i$} \\
    \vspace{\baselineskip}
    The roots of $x^3-x^2+x-1$ are $1, i,$ and $-i$
\end{center}

\subsection{Complex conjugate roots theorem}
Notice that in the above example, the complex roots of $x^3-x^2+x-1$ are $i$ and $-i$, which are complex conjugates of one another.

\begin{center}
    \noindent \fbox{For any polynomial with \textbf{real coefficients}, their complex roots will always come in \textbf{complex conjugate pairs}.}
\end{center}

\noindent So if you're given a polynomial and one of its complex roots, you can easily identify the other complex root because it will always be the conjugate. Suppose that I have a polynomial $f(x)$ of degree 5, and I was told that two of its roots are $2+i$ and $-3i$. I already know that the complex conjugates of these numbers are also roots of the same polynomial! So, I know the roots are: 

\begin{enumerate}
    \item $2+i$
    \item $2-i$
    \item $-3i$
    \item $3i$
    \item $R_5$
\end{enumerate}
where $R_5$ is the fifth root which has to be real (the polynomial has to be of degree 6 in order for $R_5$ to be complex, because then, its conjugate can exist!).

\end{document}