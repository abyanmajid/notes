\documentclass{article}

% Packages for code listing and syntax highlighting
\usepackage{listings}
\usepackage{xcolor}
\usepackage[margin=3cm]{geometry} % Adjust the margin value as desired
\usepackage{setspace}
\usepackage{tikz}

\onehalfspacing

% Define the color theme
\definecolor{codebackground}{RGB}{242, 242, 242}
\definecolor{codekeyword}{RGB}{0, 0, 255}
\definecolor{codecomment}{RGB}{63, 127, 95}
\definecolor{codestring}{RGB}{163, 21, 21}

% Code listing style for all languages
\lstdefinestyle{mystyle}{
    backgroundcolor=\color{codebackground},
    basicstyle=\footnotesize\ttfamily,
    keywordstyle=\color{codekeyword}\bfseries,
    commentstyle=\color{codecomment}\itshape,
    stringstyle=\color{codestring},
    numbers=left,
    numberstyle=\tiny\color{codecomment},
    stepnumber=1,
    numbersep=8pt,
    showstringspaces=false,
    breaklines=true,
    frame=single,
    frameround=none,
    framesep=5pt,
    rulecolor=\color{codebackground},
    tabsize=4,
    captionpos=b,
    xleftmargin=15pt,
    xrightmargin=15pt
}

% Set the default style for code listings
\lstset{style=mystyle}

% Additional packages and settings for math typesetting
\usepackage{amsmath}
\usepackage{amssymb}
\usepackage{bm}

% Define your document content
\begin{document}

\title{Complex Numbers Introduction and Cartesian Form}
\author{Abyan Majid}
\date{June 15, 2023}
\maketitle

\section{Imaginary unit $i$}

In the real number system, the square root of any negative number, such as $\sqrt{-3}$, $\sqrt{{-\frac{1}{2}}}$, and $\sqrt{-\pi}$, are not defined. Instead, mathematicians came up with the concept of an imaginary unit, denoted by $i$, which is defined by $\sqrt{-1}$.
\begin{center}
\fbox{
$i=\sqrt{-1}$
}
\end{center}

\section{Complex numbers}
Complex numbers are numbers that consist of a combination of a real part (Re) and an imaginary part (Im). In general, complex numbers can be expressed in three forms:
\begin{itemize}
    \item Cartesian form (also known as Rectangular form): \fbox{$a+bi$}
    \item Polar form (also known as Modulus-Argument form): \fbox{$r(\cos \theta + i\sin \theta)$}, often abbreviated as \fbox{$r=\text{cis}\ \theta$}
    \item Exponential form: \fbox{$re^{i\theta}$}
\end{itemize}

\subsection{Cartesian/Rectengular form}
The Cartesian/Rectangular form of complex numbers is \fbox{$a+bi$}, where $a$ is the real part, while $b$ is the imaginary part multiplied by the imaginary unit $i$.

\vspace{\baselineskip}

\noindent So, if we have a complex number $z=4+6i$, we can say that:
\begin{itemize}
    \item Re($z$) = 4
    \item Im($z$) = 6
\end{itemize}

\subsubsection{Performing arithmetic operations     in Cartesian form}
The Cartesian form \fbox{$a+bi$} is the most effective form for performing arithmetic on complex numbers. To help illustrate how to do arithmetic on complex numbers, let's suppose that we have the complex numbers $z=2+4i$ and $w=3+5i$.

\begin{itemize}
    \item \textbf{Addition and subtraction}
    
    To perform addition, simply add real parts together and imaginary parts together. Here's an example involving the complex numbers $z=2+4i$ and $w=3+5i$ we defined earlier.
    
    $z + w = (2+4i) + (3+5i)$ \\
    $z + w = (2+3) + (4+5)i$ \\
    $z + w = 5-9i $

    Likewise for subtraction; subtract real parts from real parts and imaginary parts from imaginary parts.

    $z - w = (2+4i) - (3+5i)$ \\
    $z - w = (2-3) + (4-5)i$ \\
    $z - w = -1 -1i$

    \item \textbf{Multiplication}
    
    To perform multiplication, just use the FOIL (First, Outer Inner, Last) method, where given $(a+b)(c+d)$, you follow:
    
    \begin{enumerate}
        \item F: perform $a\times c$
        \item O: perform $a\times d$
        \item I: perform $b\times c$
        \item L: perform $b\times d$
    \end{enumerate}
    So, let's now perform the FOIL method to multiply $z=2+4i$ with $w=3+5i$.

    $z \times w = (2+4i)(3+5i)$ \\
    $z \times w = 6+10i+12i+20i^2$

    Remember that $i^2$ equates to $\sqrt{-1}$. Therefore, $20i^2=20(-1)$
    
    $z \times w = -14 + 22i$

    \item \textbf{Division}
    
    We cannot divide by an imaginary number; any fraction must have a real number denominator. So, in order to divide one complex number by another, we need to introduce "\textbf{complex conjugates}".
    \begin{center}
    \rule{10cm}{0.4pt}

        \fbox{The complex conjugate of $z=a+bi$ is defined as $\bar{z}=a-bi$}
        
        To perform division between complex numbers $z=a+bi$ and $w=c+di$, you follow:
        $$\frac{z}{w}=\frac{z}{w} \times \frac{\bar{w}}{\bar{w}}$$
        $$\frac{z}{w}=\frac{(a+bi)(c-di)}{(c+di)(c-di)}$$
    
    \rule{10cm}{0.4pt}  
    \end{center}

    A \textbf{complex conjugate} of a complex number is the number with the imaginary part negated.
    Complex conjugates are useful for divison between
    complex numbers because a complex number multiplied
    by its conjugate always results in a real number. 
    
    $$(1+2i)(1-2i)=1-2i+2i-4i^2=\fbox{5}$$

    Recall that a divisor must always be a real number, so in the
    context of complex number divison, we can theoretically multiply
    the divisor with a complex conjugate to turn it into a real number! So, let's now
    perform divison between $z=2+4i$ and $w=3+5i$,

    $$\frac{z}{w}=\frac{(2+4i)(3-5i)}{(3+5i)(3-5i)}$$
    $$=\frac{6-10i+12i-20i^2}{9-15i+15i-25i^2}$$
    $$=\frac{26+2i}{34}$$
    $$\approx 0.765+0.0588i \phantom{1} \text{(3 sf.)}$$

\end{itemize}

\subsubsection{Argand diagram (Complex plane)}

The Argand diagram, otherwise known as the complex plane, consists of a horizontal real axis (Re) and a vertical imaginary axis (Im). It works like your usual Cartesian plane. Let's say that we want to plot a complex number $z=2+3i$. To do that, just go 2 units to the right from the origin along the real axis, and then go 3 units up.

\begin{center}
    \begin{tikzpicture}[scale=0.35]
        % Draw axes
        \draw[->] (-10, 0) -- (10, 0) node[right] {Re};
        \draw[->] (0, -10) -- (0, 10) node[above] {Im};
        
        % Labels for the ends of the axes
        \node at (-9.5, -1) {$-10$};
        \node at (9.5, -1) {$10$};
        \node at (-1.35, -9.5) {$-10$};
        \node at (-1, 9.5) {$10$};
        \node at (2, -1) {2};
        \node at (-1, 3) {3};

        % Draw dashed line
        \draw[dashed, red, line width=1.75pt] (0, 3) -- (2, 3);
        \draw[dashed, red, line width=1.75pt] (2, 0) -- (2, 3);

        % Plot complex number z
        \draw[fill=black] (2, 3) circle[radius=7pt] node[above right] {$z=2+3i$};
    \end{tikzpicture}
\end{center}

\subsubsection{Modulus/absolute value}
Modulus, also known as "absolute value", is denoted as can be defined as:
\begin{itemize}
    \item $|x|$, the distance from zero to a real number $x$
    \item or $|z|$, the distance from the origin to a complex number $|z=a+bi|$
\end{itemize}

\noindent$x$ and $z$ are generic variables and can be replaced by any other name.

\vspace{\baselineskip}

\noindent\textbf{Modulus of a real number} \\
For any real number $x$, the modulus is always $x$ (eg. it takes 3 units to go from 0 to 3). Alternatively, on the "real number line", the modulus of any real number $x$ is:
\begin{itemize}
    \item $x$ if $x$ is positive
    \item $-x$ if $x$ is negative
\end{itemize}

\noindent\textbf{Modulus of a complex number} \\
For any complex number $z=a+bi$, the modulus is \fbox{$|z|=\sqrt{a^2+b^2}$}. You might realize that this is perfectly identical to Pythagoras' Theorem, and that's because it is (See the following argand diagram)

\begin{center}
    \begin{tikzpicture}[scale=0.35]
        % Draw axes
        \draw[->] (-10, 0) -- (10, 0) node[right] {Re};
        \draw[->] (0, -10) -- (0, 10) node[above] {Im};
        
        % Labels for the ends of the axes
        \node at (-9.5, -1) {$-10$};
        \node at (9.5, -1) {$10$};
        \node at (-1.35, -9.5) {$-10$};
        \node at (-1, 9.5) {$10$};
        \node at (3.5, 4) {$|z|$};
        \node at (8, 2.5) {$b$};
        \node at (3.5, -1) {$a$};

        % Draw dashed line
        \draw[dashed, red, line width=1.75pt] (0, 0) -- (7, 0);
        \draw[dashed, red, line width=1.75pt] (7, 0) -- (7, 5);
        \draw[dashed, blue, line width=1.75pt] (0, 0) -- (7, 5);

        % Plot complex number z
        \draw[fill=black] (7, 5) circle[radius=7pt] node[above right] {$z=7+5i$};
    \end{tikzpicture}

In this case, we can find the modulus of $z=7+5i$ like so:

$$ |z|=\sqrt{7^2+5^2} $$
$$ |z|= \sqrt{74} \approx 8.60 \text{units (3 sf.)}$$

\end{center}

\end{document}