\documentclass{article}

% Packages for code listing and syntax highlighting
\usepackage{listings}
\usepackage{xcolor}
\usepackage[margin=3cm]{geometry} % Adjust the margin value as desired
\usepackage{setspace}
\usepackage{tikz}
\usepackage{graphicx}
\usepackage{float}
\usepackage{textcomp}
\usepackage{multicol}
\usepackage{enumitem}

\onehalfspacing

% Define the color theme
\definecolor{codebackground}{RGB}{242, 242, 242}
\definecolor{codekeyword}{RGB}{0, 0, 255}
\definecolor{codecomment}{RGB}{63, 127, 95}
\definecolor{codestring}{RGB}{163, 21, 21}

% Code listing style for all languages
\lstdefinestyle{mystyle}{
    backgroundcolor=\color{codebackground},
    basicstyle=\footnotesize\ttfamily,
    keywordstyle=\color{codekeyword}\bfseries,
    commentstyle=\color{codecomment}\itshape,
    stringstyle=\color{codestring},
    numbers=left,
    numberstyle=\tiny\color{codecomment},
    stepnumber=1,
    numbersep=8pt,
    showstringspaces=false,
    breaklines=true,
    frame=single,
    frameround=none,
    framesep=5pt,
    rulecolor=\color{codebackground},
    tabsize=4,
    captionpos=b,
    xleftmargin=15pt,
    xrightmargin=15pt
}

% Set the default style for code listings
\lstset{style=mystyle}

% Additional packages and settings for math typesetting
\usepackage{amsmath}
\usepackage{amssymb}
\usepackage{bm}

% Define your document content
\begin{document}

\title{Encoding Sets of Symbols in Binary Logic}
\author{Abyan Majid}
\date{July 6, 2023}
\maketitle

\section{Encoding sets of symbols}

To encode a set of symbols you must:
\begin{enumerate}
    \item Derive $n$, which is the minimum amount of bits to encode all symbols in the set.
    \begin{center}
        $n$ bits $\rightarrow 2^n$ combinations 
    \end{center}
    Recall that given a set $S$, $|S|$ is the cardinality of the set and is a measure of how much elements are in the set. So, you should get a value of $n$ that allows a number of combinations that is greater than or equal to $|S|$, the number of elements.
    $$2^n\geq|S|$$
    \begin{center}
        Now, to get $n$, you make use of the logarithmic identity: $\log_a b=c \longleftrightarrow a^c=b$. In our context, we can specifically do (and this is what you should remember):
        \[
        \fbox{$n = \lceil \log_2 |S| \rceil$}
        \]
        Here, $\log_2 |S|$ gives the "exact" number of bits needed for encoding $S$. But, since it's invalid to have a floating-point number of bits, we use the ceiling function "$\lceil x \rceil$" to always round up to the nearest integer.
    \end{center}
    \begin{center}
        FOR EXAMPLE: Suppose we have $CLASS=\{Storm, Fire, Ice, Death, Life, Myth, Balance\}$. Given that $|CLASS|=7$, we have:
        $$n=\lceil log_2 (7) \rceil=\fbox{3}$$
        We know that "3" is indeed the minimum amount of bits needed to encode the symbols of $CLASS$ because:
        $$2^3=8\geq|S|=7$$
    \end{center}
    \item Assign a unique binary of $n$ bits for each symbol where necessary.
    \begin{center}
        This step should be pretty easy. Following from our previous example, we can assign a unique 3-bit binary for each symbol in $CLASS$. Of course there is no right order, but here's one way to do it: \\
        \vspace{\baselineskip}
        \begin{tabular}{|c|c|}
            \hline
            Symbol & 3-Bit Binary \\
            \hline
            Storm & 000 \\
            Fire & 001 \\
            Ice & 010 \\
            Death & 011 \\
            Life & 100 \\
            Myth & 101 \\
            Balance & 110 \\
            \hline
          \end{tabular} \\
          \vspace{\baselineskip}
          Since a 3-bit binary system has 8 combinations ($2^3$), and $CLASS$ has only 7 elements, we do not have any symbol to assign the leftover binary (i.e. 111) - and that's perfectly okay!
    \end{center}
\end{enumerate}

\end{document}